\documentclass[]{article}
\usepackage[T1]{fontenc}
\usepackage{lmodern}
\usepackage{amssymb,amsmath}
\usepackage{ifxetex,ifluatex}
\usepackage{fixltx2e} % provides \textsubscript
% Set line spacing
% use upquote if available, for straight quotes in verbatim environments
\IfFileExists{upquote.sty}{\usepackage{upquote}}{}
\ifnum 0\ifxetex 1\fi\ifluatex 1\fi=0 % if pdftex
  \usepackage[utf8]{inputenc}
\else % if luatex or xelatex
  \ifxetex
    \usepackage{mathspec}
    \usepackage{xltxtra,xunicode}
  \else
    \usepackage{fontspec}
  \fi
  \defaultfontfeatures{Mapping=tex-text,Scale=MatchLowercase}
  \newcommand{\euro}{€}
\fi
% use microtype if available
\IfFileExists{microtype.sty}{\usepackage{microtype}}{}
\usepackage[margin=1in]{geometry}
\usepackage{graphicx}
% Redefine \includegraphics so that, unless explicit options are
% given, the image width will not exceed the width of the page.
% Images get their normal width if they fit onto the page, but
% are scaled down if they would overflow the margins.
\makeatletter
\def\ScaleIfNeeded{%
  \ifdim\Gin@nat@width>\linewidth
    \linewidth
  \else
    \Gin@nat@width
  \fi
}
\makeatother
\let\Oldincludegraphics\includegraphics
{%
 \catcode`\@=11\relax%
 \gdef\includegraphics{\@ifnextchar[{\Oldincludegraphics}{\Oldincludegraphics[width=\ScaleIfNeeded]}}%
}%
\ifxetex
  \usepackage[setpagesize=false, % page size defined by xetex
              unicode=false, % unicode breaks when used with xetex
              xetex]{hyperref}
\else
  \usepackage[unicode=true]{hyperref}
\fi
\hypersetup{breaklinks=true,
            bookmarks=true,
            pdfauthor={Vishal Sood},
            pdftitle={Regression Trees and Rule Based Models},
            colorlinks=true,
            citecolor=blue,
            urlcolor=blue,
            linkcolor=magenta,
            pdfborder={0 0 0}}
\urlstyle{same}  % don't use monospace font for urls
\setlength{\parindent}{0pt}
\setlength{\parskip}{6pt plus 2pt minus 1pt}
\setlength{\emergencystretch}{3em}  % prevent overfull lines
\setcounter{secnumdepth}{0}

%%% Change title format to be more compact
\usepackage{titling}
\setlength{\droptitle}{-2em}
  \title{Regression Trees and Rule Based Models}
  \pretitle{\vspace{\droptitle}\centering\huge}
  \posttitle{\par}
  \author{Vishal Sood}
  \preauthor{\centering\large\emph}
  \postauthor{\par}
  \predate{\centering\large\emph}
  \postdate{\par}
  \date{13 May, 2015}




\begin{document}

\maketitle


\section{Example Data}\label{example-data}

We will use the solubility data for illustrations. For cross-validation
we need to set the folds, and a control to tell \emph{caret::train} how
to train the model. For creating data partitions caret provides
functions,

\begin{enumerate}
\def\labelenumi{\arabic{enumi}.}
\itemsep1pt\parskip0pt\parsep0pt
\item
  \emph{createDataPartition}: a series of test/training partitions
\item
  \emph{createResample}: one or more bootstrap samples
\item
  \emph{createFolds}: splits the data into k groups
\item
  \emph{createTimeSlices}: creates cross-validation sample information
  to be used with time series data
\end{enumerate}

To create the CV folds we will use \emph{createFolds}, that creates 10
folds by default, returning the data as a list of vectors containing the
sample positions corresponding to the data used during training.
Remember that having 10 folds means that 1 of the ten folds will be held
out from training for validation. The returned vectors will then contain
90\% of the samples. This list will then be passed to
\emph{trainControl} to obtain a \emph{control} object to be used with
\emph{caret::train}.

\section{Basic Regression Trees}\label{basic-regression-trees}

Packages for regression trees:

\begin{enumerate}
\def\labelenumi{\arabic{enumi}.}
\itemsep1pt\parskip0pt\parsep0pt
\item
  \textbf{rpart} splits using CART methodology, the \emph{rpart}
  function, two commonly used control parameters that are used in
  training are the complexity parameter (cp) and maximum node depth
  (maxdepth), both accessible in \emph{caret::train}. For \emph{cp} set
  \emph{method=``rpart''}, for maxdepth, *method=``rpart2''.
\item
  \textbf{party} splits using conditional inference framework, the
  \emph{ctree} function.
\end{enumerate}

We make two CART models,

Tune the model

\begin{verbatim}
## Warning in nominalTrainWorkflow(x = x, y = y, wts = weights, info =
## trainInfo, : There were missing values in resampled performance measures.
\end{verbatim}

CART

951 samples 228 predictors

No pre-processing Resampling: Cross-Validated (10 fold)

Summary of sample sizes: 856, 857, 855, 856, 856, 855, \ldots{}

Resampling results across tuning parameters:

cp RMSE Rsquared RMSE SD Rsquared SD 0.002899268 0.9716246 0.7752447
0.06541238 0.03897881 0.002992913 0.9660816 0.7779183 0.05330954
0.03562041 0.003284296 0.9767689 0.7727573 0.04958512 0.03322272
0.003560157 0.9864913 0.7681005 0.04915569 0.03456546 0.003857172
0.9978151 0.7627426 0.05749370 0.03635197 0.004007488 1.0005677
0.7614942 0.05130042 0.03486357 0.004050192 1.0080565 0.7573802
0.05214459 0.03849504 0.004224784 1.0154052 0.7534214 0.05669100
0.04255321 0.004427556 1.0174706 0.7524694 0.05439656 0.04070176
0.004737500 1.0196066 0.7514764 0.05532976 0.04411784 0.005204160
1.0230892 0.7499148 0.05409936 0.04451837 0.006181580 1.0207997
0.7495512 0.05958220 0.04837953 0.006518655 1.0308117 0.7448964
0.05745456 0.04726132 0.008288491 1.0464044 0.7369148 0.06873592
0.05097230 0.008861298 1.0482274 0.7357991 0.06448159 0.05027666
0.011501953 1.0951525 0.7125953 0.08962476 0.05412205 0.011781343
1.1017983 0.7085687 0.09360527 0.05754522 0.015535229 1.1218072
0.6972460 0.09949073 0.05991353 0.017892740 1.1317681 0.6906429
0.10606249 0.06810422 0.026503011 1.1747433 0.6680776 0.08700156
0.05948686 0.047291114 1.2585476 0.6174856 0.10318820 0.07279270
0.061802695 1.3197449 0.5802679 0.12398650 0.08602437 0.069715098
1.4025314 0.5265519 0.12300733 0.08459634 0.137700138 1.5373482
0.4375590 0.19960327 0.09333503 0.373005064 1.9564323 0.2781173
0.19891634 0.00920422

RMSE was used to select the optimal model using the smallest value. The
final value used for the model was cp = 0.002992913. And plot the tuning
\includegraphics{treeModels_files/figure-latex/plotcartune-1.pdf}
Variable importance

rpart variable importance

only 20 most important variables shown (out of 228)

\begin{verbatim}
              Overall
\end{verbatim}

NumNonHBonds 0.9070 SurfaceArea2 0.7564 NumMultBonds 0.7453 NumCarbon
0.7257 MolWeight 0.6693 NumNonHAtoms 0.6093 NumBonds 0.5871 FP116 0.4788
NumOxygen 0.4706 NumRotBonds 0.3794 SurfaceArea1 0.3554 NumHydrogen
0.3522 FP081 0.2594 FP075 0.2435 FP077 0.1992 HydrophilicFactor 0.1164
FP203 0.0000 NumRings 0.0000 FP107 0.0000 FP053 0.0000 Test the model on
the test-data

Conditional inference tree

Conditional Inference Tree

951 samples 228 predictors

No pre-processing Resampling: Cross-Validated (10 fold)

Summary of sample sizes: 856, 857, 855, 856, 856, 855, \ldots{}

Resampling results across tuning parameters:

mincriterion RMSE Rsquared RMSE SD Rsquared SD 0.7500000 0.9269810
0.7903849 0.1072444 0.06710406 0.7766667 0.9260851 0.7905803 0.1103460
0.06810741 0.8033333 0.9271195 0.7901073 0.1108398 0.06829510 0.8300000
0.9277246 0.7906922 0.1049071 0.06333440 0.8566667 0.9254289 0.7913914
0.1115011 0.06598342 0.8833333 0.9305731 0.7892532 0.1120570 0.06569786
0.9100000 0.9312221 0.7891169 0.1066836 0.06480620 0.9366667 0.9376111
0.7860495 0.1047474 0.06470085 0.9500000 0.9403788 0.7847694 0.1038905
0.06301081 0.9633333 0.9491457 0.7806482 0.1075643 0.06510664 0.9900000
0.9574527 0.7767184 0.1105580 0.07000131

RMSE was used to select the optimal model using the smallest value. The
final value used for the model was mincriterion = 0.8566667.
\includegraphics{treeModels_files/figure-latex/tunecondinftr-1.pdf}

\section{Regression Model Trees}\label{regression-model-trees}

\section{Rule-Based Models}\label{rule-based-models}

\section{Bagged Trees}\label{bagged-trees}

\section{Random Forests}\label{random-forests}

\textbf{Tree Correlation} (Hastie et al 2008): If we start with a
sufficiently large number of original samples and a relationship between
predictors and response that can be adequately modeled by a tree, then
trees from different bootstrap samples may have similar structures to
each other ( especially at the top of the trees) due to the underlying
relationship. This can prevent bagging from optimally reducing variance
of the predicted values.

So variance can be further reduced, Breiman's Random Forest:

\begin{enumerate}
\def\labelenumi{\arabic{enumi}.}
\item
  m \textless{}- NumberOfModelsToBuild
\item
  for i = 1 to m do
\item
  Generate a bootstrap sample of the original data
\item
  Train a tree model on this sample
\item
  for each split do
\item
\begin{verbatim}
Randomly select k (< P) of the original predictors
\end{verbatim}
\item
\begin{verbatim}
Select the best predictor among the k predictors and partition the data
\end{verbatim}
\item
  end
\item
  Use typical tree model stopping critera to determine when a tree is
  complete (no pruning)
\item
  end
\end{enumerate}

Tuning parameter: number of models to build.

Breiman: linear combination of many independent learners reduces the
variance of the overall ensemble relative to any individual learners.
Random forest achieves this by selecting strong, complex learners that
exhibit low bias. Robust to noisy response, however the independence of
learners can underfit data when the response is not noisy.

CART or conditional inference trees can be used as the base learner.
Tuning parameter does not have a drastic effect on performance.
Cross-validation RMSE can be very similar to out-of-bag error estimate.

Can we understand the relationship between the predictors and the
response?

Breiman: Randomly permute the values of each predictor for the oob
sample of one predictor at a time for each tree. The difference in
predictive performance between the non-permuted sample and the permuted
sample for each predictor is recorded and aggregated across the entire
forest.

Or measure the improvement in node purity based on the performance
metric for each predictor at each occurence of that predictor across the
forest, aggregate across the forest to determine the overall importance
for the predictor.

Problems with collinearty in determining predictor importances.
Dillution effects.

\section{Boosting}\label{boosting}

Originally for classification. From AdaBoost for classification to
Friedman's stochastic gradient boosting machine. From AdaBoost for
classification to Friedman's stochastic gradient boosting machine. From
AdaBoost for classification to Friedman's stochastic gradient boosting
machine.

1990s, Schapire 1990, 1999; Freund 1995, influence by learning theory.
Weak classifiers are combined (boosted) to produce an ensemble
classifier with a superior generalized misclassification error rate.
AdaBoost is implementation (Schapire 1999)

Used widely with applications in gene expression (Dudoit 2002, Ben-Dor
2000), chemometrics (Varmuze 2003), music genre identification (Bergstra
2006).

(Friedman 2000): AdaBoost to statistical concepts of loss functions,
additive modeling, and logistic regression. Boosting can be interpreted
as a forward stagewise additive model that minimizes exponential loss.
Resulted in a simple, elegant, and highly adaptable algorithm for
different kinds of problems (Friedman 2001): gradient boosting machines.
Both regression, classification.

\textbf{Gradient Boosting Machines}: given a loss function, and a weak
learner, finds an additive model that minimized the loss function.
Initialize with the best guess of the response. Calculate the gradient
(eg residual), fit a model to the residuals to minimize the loss. Add
the current model to the previous model. Iterate.

With trees as base learners, two tuning parameters: tree depth
(interaction depth), number of iterations. Tree depth is also called
interaction depth: think of each subsequential split as a higher-level
interaction term with all of the other previous split predictors.

With squared error as the loss function, . let D = treeDepth . let K =
numIterations . let L = learningRate . iteration(observed, currentPred)
= currentPred + L*updatePred(observed, currentPred) .
updatePred(residuals currentPred) = predict(T) where T =
regressionTree(depth=D, response=residuals(observed, currentPred))

Difference from random forests: dependent on past trees, have minimum
depth, and contribute unequally to the final model.

Can be susceptible to over-fitting. Learner optimally fits the gradient,
so Boosting will select the optimal learner at each stage of the
algorithm. The greedy strategy may not find the optimal global model,
and may over-fit the training data. So regularization through
\(\lambda\), the learning rate. (Ridgeway 2007)

Friedman's \textbf{stochastic gradient boosting}: update using a
randomly sampled fraction of the training data ( around 0.5), the
fraction becomes another tuning parameter.

Variable importance: function of the reduction in squared error -- due
to each predictor is summed within each tree in the ensemble. Averaged

\end{document}
